\section*{Лекция №2: Анализ устойчивости численных схем}
\addcontentsline{toc}{section}{Лекция №2}

\subsection*{Введение}
\addcontentsline{toc}{subsection}{Введение}

В данной лекции мы рассмотрим понятие устойчивости численных методов. Устойчивость является ключевым аспектом при решении дифференциальных уравнений численными методами, поскольку она определяет, насколько надежным будет полученное численное решение при малых изменениях начальных условий или параметров задачи.

\subsection*{Основные понятия и определения}
\addcontentsline{toc}{subsection}{Основные понятия и определения}


Очевидно, что при аппроксимации задачи особое внимание следует обратить на запись в разностном виде начального условия для производной $\frac{\partial u}{\partial t}$.

Пусть дана равномерная по $x$ и $t$ сетка $\omega_{h \tau}$ с шагами $h$ и $\tau$. Если мы воспользуемся простейшей аппроксимацией
$$
u_t(x, 0)=\bar{u}_0(x),
$$

то погрешность аппроксимации будет оцениваться $O(\tau)$. Представим $u_t(f, 0)$ в виде
$$
u_t(x, 0)=\frac{u(x, \tau)-u(x, 0)}{\tau}=\frac{\partial u(x, 0)}{\partial t}+\frac{\tau}{2} \frac{\partial^2 u(x, 0)}{\partial t^2}+O\left(\tau^2\right) .
$$

Обратимся теперь к исходному дифференциальному уравнению и найдем
$$
\frac{\partial^2 u(x, 0)}{\partial t^2}=\frac{\partial^2 u(x, 0)}{\partial x^2}+f(x, 0)=L u_0(x)+f(x, 0), \quad L u_0=\frac{d^2 u_0}{d x^2},
$$

так как $\frac{\partial^2 u(x, 0)}{\partial x^2}=\frac{d^2 u_0(x)}{d x^2} \cdot$ Отсюда следует, что
$$
u_t(x, 0)-0,5 \tau\left(L u_0+f(x, 0)\right)=\frac{\partial u(x, 0)}{\partial t}+O\left(\tau^2\right) .
$$

Поэтому разностное начальное условие $y_t(x, 0)=\tilde{u}_0(x)$, где $\tilde{u}_0(x)=\bar{u}_0(x)+0,5 \tau\left(L u_0+f(x, 0)\right)$, аппроксимирует на решении задачи условие $\frac{\partial u(x, 0)}{\partial t}=\bar{u}_0(x)$ со вторым порядком по $\tau$.

Условие $u(x, 0)=u_0(x)$ и краевые условия в данном случае аппроксимируются точно. В качестве разностной аппроксимации уравнения можно взять, например, одну из схем.

Из предыдущего изложения следует, что при повышевии порядка аппроксимации краевых и начальных условий мы использовали существование и непрерывность производных, входящих в уравнение, на границе области (при $x=0$ или $t=0$ ), а также существование и ограниченность третьих производных решения.

\textbf{Пример}. Трехслойная разностная схема для уравнения теплопроводности. Рассмотрим первую краевую задачу
$$
\begin{aligned}
	\frac{\partial u}{\partial t} & =\frac{\partial^2 u}{\partial x^2}+f(x, t), \quad 0<x<1, \quad 0<t \leqslant t_0, \\
	u(x, 0) & =u_0(x), \quad u(0, t)=u_1(t), \quad u(1, t)=u_2(t) .
\end{aligned}
$$

Для решения уравнения теплопроводности часто применяются так называемые трехслойные схемы, использующие значения сеточной функции $y^{j-1}(x), y^j(x), y^{j+1}(x)$ на трех временных слоях $t_{j-1}, t_j, t_{j+1}$.

Например, трехслойная симметричная схема на равномерной сетке $\omega_{h г}$ с шагами $h$ и $\tau$ выглядит следующим образом:
$$
\begin{gathered}
	\frac{y^{j+1}-y^{j-1}}{2 \tau}=\Lambda\left(\sigma y^{j+1}+(1-2 \sigma) y^j+\sigma y^{j-1}\right)+\varphi^j, \\
	y_i^0=u_0\left(x_i\right), \quad y_0^j=u_1^j, \quad y_N^j=u_2^j,
\end{gathered}
$$

где $\Lambda y=y_{\overline{x} x}, \sigma$ - вещественный параметр, $\varphi^j=f\left(x_i, t_j\right)$.
Так как центральная разностная производная по $t$ аппроксимирует $\left.\frac{\partial u}{\partial t}\right|_{t=t_j}$ со вторым порядком по $\tau$, а $\Lambda u=\frac{\partial^2 u}{\partial x^2}+O\left(h^2\right)_x$ то схема аппроксимирует уравнение с $O\left(h^2+\tau^2\right)$. Heтрудно, однако, заметить, что задача не доопределена. Для применения трехслойной схемы требуется задать еще одно начальное условие, например, задать $y(x, t)$ на первом слое. Естественно потребовать, чтобы введение этого условия сохраняло аппроксимацию $O\left(\tau^2+h^2\right)$.

Можно указать два способа задания $y(x, \tau)$. Первый способ состоит в том, что мы делаем первый шаг по двухслойной схеме
$$
\frac{y^1-y^0}{\tau}=\frac{1}{2} \Lambda\left(y^1+y^0\right)+\varphi^0,
$$

обеспечивающей определение $y(x, \tau)$ с точностью $O\left(\tau^2+h^2\right)$. Второй способ состоит в том, что мы ищем значение $y(x, \tau)$ в виде $y(x, \tau)=u_0(x)+\tau \mu(x)$ п подбираем $\mu$ так, чтобы погрешность $y(x, \tau)-u(x, \tau)$ не превосходила $O\left(\tau^2+h^2\right)$. Подставим в формулу
$$
u(x, \tau)-u_0(x)=\left.\tau \frac{\partial u}{\partial t}\right|_{t=0}+\left.\frac{\tau^2}{2} \frac{\partial^2 u}{\partial t^2}\right|_{t=0}+O\left(\tau^3\right):
$$

значение $\left.\frac{\partial u}{\partial t}\right|_{t=0}$ исходя из дифференциального уравнения
$$
\left.\frac{\partial u}{\partial t}\right|_{t=0}=L u_0+f(x, 0), \quad L u_0=\frac{d^2 u_0}{d x^2} .
$$

Тогда получим $\mu=L u_0+f(x, 0)$, и, следовательно,
$$
y(x, \tau)=u_0(x)+\tau\left(u_0^{\prime \prime}(x)+f(x, 0)\right) .
$$

\subsubsection*{Последовательность операторов и устойчивость}
\addcontentsline{toc}{subsubsection}{Последовательность операторов и устойчивость}

Рассмотрим последовательность операторов \(A\), которая играет ключевую роль в анализе устойчивости. Устойчивость системы определяется её способностью сохранять определенные характеристики и поведение в ответ на внешние и внутренние возмущения.

\subsubsection*{Самосопряженные операторы}
\addcontentsline{toc}{subsubsection}{Самосопряженные операторы}

Самосопряженный оператор \(A\) на комплексном гильбертовом пространстве определяется условием \(\langle Ax, y \rangle = \langle x, Ay \rangle\). Это условие указывает на то, что оператор и его сопряженный оператор действуют одинаково.

\subsubsection*{Собственные значения и векторы}
\addcontentsline{toc}{subsubsection}{Собственные значения и векторы}

Собственный вектор линейного оператора — это вектор, который при умножении на оператор изменяется только по величине. Собственное значение соответствует этому коэффициенту изменения. Для самосопряженного оператора все собственные значения являются действительными.


\subsection*{Анализ устойчивости}
\addcontentsline{toc}{subsection}{Анализ устойчивости}

Начнем разговор про устойчивость. У нас имеется последовательность, оператор $A_h u_h = f_h$. Тогда разница между точным решением и есть исходное уравнение $Au = f$. Если мы вычтем теперь точное уравнение из исходного, то мы получим условие $A_h (u_h - u_h^T) = r_h$ , где $u_h^T$ - сеточная функция, которая соответствует значениям точного решения, $r_h = A_h u_h^T - f_h$. Тогда разница $z_h = u_h - u_h^T$ удовлетворяет уравнению $z_h = A^{-1} r_h$. Это все верно для линейных уравнений (т.к. для нелинейных мы вычитать одно из другого не можем, вернее можем, но получится все по другому). Тогда если норма оператора ограничена $||A^{-1}_h|| \leq M$, то тогда мы получим оценку погрешности приближенного и точного решения через величину невязки $r_h$.

\begin{definition}
	Система называется устойчивой, все разностные схемы называются устойчивыми, если у нас исполнено условие:
	
	\begin{equation}
		\label{eq:cond-stable}
		||A^{-1}_h|| \leq M
	\end{equation}	

\end{definition}

\begin{mdframed}
	Если аппроксимация характеризует связь численного метода с исходным дифференциальным уравнением, то устойчивость внутренним свойством вычислительного метода. Оно не связанно с самим уравнением.Оно описывает лишь свойства разностных операторов.
\end{mdframed}

Мы должны оценить норму $||A^{-1}_h||$.

Но прежде чем мы это сделаем поговорим о самосопряженных операторах. Самосопряженный оператор подразумевает, что у нас в пространстве имеется скалярное произведение, то есть содержится пространство кон функций, которые являются Гильбертовыми пространствами. Самосопряженные операторы имеют ортогональные базисы собственных векторов.


\begin{mdframed}
	Приведем пример того, что такое собственный вектор линейного оператора: Если мы возьмем, например, качели и начнем рукой их раскачивать. Можем качать как угодно. Такие колебания называются \textbf{вынужденными}, а если поднимем вверх и отпустим, то такие колебания называются \textbf{собственными}. Теперь если посмотреть уравнение динамики, то задача нахождения собственных колебаний связана с задачей вычисления некоторого оператора $Ax = \lambda x$. Вот такие уравнения они позволяют найти собственные колебания. Вектор $x$ ненулевой, называется собственным \textbf{вектором} оператора $A$, $\lambda$ его собственным значением.
\end{mdframed}

Покажем, что два собственных вектора, которые соответствуют разным собственным значениям ортогональны друг другу.

\begin{theorem}
	Два собственных вектора, которые соответствуют разным собственным значениям ортогональны друг другу.
\end{theorem}

\begin{proof}
	Пусть имеется два вектора $x$ и $y$, которые являются собственными векторами самосопряженного оператора $A_h$. 
	
	\begin{align}
		A_x = \lambda_1 x \label{eq:system1.1}\\
		A_y = \lambda_2 y \label{eq:system1.2}
	\end{align}
	
	Если мы теперь скалярно умножим \eqref{eq:system1.1} и \eqref{eq:system1.2}:
	
	\begin{align}
		(A_x,y) = \lambda_1 (x, y) \label{eq:system2.1}\\
		(A_y,x) = \lambda_2 (x, y) \label{eq:system2.2}
	\end{align}
	
	Вычтем теперь из \eqref{eq:system2.1} \eqref{eq:system2.2}:
	
	\begin{equation}
		\label{eq:diff-system}
		0 = (\lambda_1 - \lambda_2)(x, y)
	\end{equation}
	
	Но по условию $\lambda_1 \neq \lambda_2$, тогда $(x, y) = 0$.
	
	Используя тот факт, что самосопряженный оператор обладает базисом из собственных векторов, то можно сделать вывод, что любой вектор $z$, принадлежащий нашему пространству, может быть разложен по элементам этого базиса: $z = \sum\limits_i z_i x_i$, где $x_i$ является собственным вектором матрицы $A$. Тогда $Az = \sum\limits_i z_i x_i \lambda i$. Тогда:
	
	\begin{align}
		||A_z|| & = \sqrt{(A_z, A_z)} \label{eq:norm-az}\\
		(A_z, A_z) & = \sum\limits_i z_i^2 \lambda_i^2 = \sum\limits_i z_i^2\lambda_{\min}^2 = \lambda_{\max}^2 ||z||^2 \label{eq:scalar-mul-1}
	\end{align}
	
	А норма $||z|| = (z, z) = \sum\limits_i z_i^2$. Тогда справедливо утверждение:
	
	\begin{equation}
		\label{eq:norm-az-leq}
		||A_z|| \leq |\lambda_{\max}| ||z||
	\end{equation}
	
	
	

	
\end{proof}

Рассмотрим случай самосопряженного оператора $A_h$. Пусть $A_h = A^{*}_n$, тогда норма самосопряженного оператора 