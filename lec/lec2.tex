\section*{Лекция №2: Анализ устойчивости численных схем}
\addcontentsline{toc}{section}{Лекция №2}

Начнем разговор про устойчивость. У нас имеется последовательность, оператор $A_h u_h = f_h$. Тогда разница между точным решением и есть исходное уравнение $Au = f$. Если мы вычтем теперь точное уравнение из исходного, то мы получим условие $A_h (u_h - u_h^T) = r_h$ , где $u_h^T$ - сеточная функция, которая соответствует значениям точного решения, $r_h = A_h u_h^T - f_h$. Тогда разница $z_h = u_h - u_h^T$ удовлетворяет уравнению $z_h = A^{-1} r_h$. Это все верно для линейных уравнений (т.к. для нелинейных мы вычитать одно из другого не можем, вернее можем, но получится все по другому). Тогда если норма оператора ограничена $||A^{-1}_h|| \leq M$, то тогда мы получим оценку погрешности приближенного и точного решения через величину невязки $r_h$.

\begin{definition}
	Система называется устойчивой, все разностные схемы называются устойчивыми, если у нас исполнено условие:
	
	\begin{equation}
		\label{eq:cond-stable}
		||A^{-1}_h|| \leq M
	\end{equation}	

\end{definition}

\begin{mdframed}
	Если аппроксимация характеризует связь численного метода с исходным дифференциальным уравнением, то устойчивость внутренним свойством вычислительного метода. Оно не связанно с самим уравнением.Оно описывает лишь свойства разностных операторов.
\end{mdframed}

Мы должны оценить норму $||A^{-1}_h||$.

Но прежде чем мы это сделаем поговорим о самосопряженных операторах. Самосопряженный оператор подразумевает, что у нас в пространстве имеется скалярное произведение, то есть содержится пространство кон функций, которые являются Гильбертовыми пространствами. Самосопряженные операторы имеют ортогональные базисы собственных векторов.


\begin{mdframed}
	Приведем пример того, что такое собственный вектор линейного оператора: Если мы возьмем, например, качели и начнем рукой их раскачивать. Можем качать как угодно. Такие колебания называются \textbf{вынужденными}, а если поднимем вверх и отпустим, то такие колебания называются \textbf{собственными}. Теперь если посмотреть уравнение динамики, то задача нахождения собственных колебаний связана с задачей вычисления некоторого оператора $Ax = \lambda x$. Вот такие уравнения они позволяют найти собственные колебания. Вектор $x$ ненулевой, называется собственным \textbf{вектором} оператора $A$, $\lambda$ его собственным значением.
\end{mdframed}

Покажем, что два собственных вектора, которые соответствуют разным собственным значениям ортогональны друг другу.

\begin{theorem}
	Два собственных вектора, которые соответствуют разным собственным значениям ортогональны друг другу.
\end{theorem}

\begin{proof}
	Пусть имеется два вектора $x$ и $y$, которые являются собственными векторами самосопряженного оператора $A_h$. 
	
	\begin{align}
		A_x = \lambda_1 x \label{eq:system1.1}\\
		A_y = \lambda_2 y \label{eq:system1.2}
	\end{align}
	
	Если мы теперь скалярно умножим \eqref{eq:system1.1} и \eqref{eq:system1.2}:
	
	\begin{align}
		(A_x,y) = \lambda_1 (x, y) \label{eq:system2.1}\\
		(A_y,x) = \lambda_2 (x, y) \label{eq:system2.2}
	\end{align}
	
	Вычтем теперь из \eqref{eq:system2.1} \eqref{eq:system2.2}:
	
	\begin{equation}
		\label{eq:diff-system}
		0 = (\lambda_1 - \lambda_2)(x, y)
	\end{equation}
	
	Но по условию $\lambda_1 \neq \lambda_2$, тогда $(x, y) = 0$.
	
	Используя тот факт, что самосопряженный оператор обладает базисом из собственных векторов, то можно сделать вывод, что любой вектор $z$, принадлежащий нашему пространству, может быть разложен по элементам этого базиса: $z = \sum\limits_i z_i x_i$, где $x_i$ является собственным вектором матрицы $A$. Тогда $Az = \sum\limits_i z_i x_i \lambda i$. Тогда:
	
	\begin{align}
		||A_z|| & = \sqrt{(A_z, A_z)} \label{eq:norm-az}\\
		(A_z, A_z) & = \sum\limits_i z_i^2 \lambda_i^2 = \sum\limits_i z_i^2\lambda_{\min}^2 = \lambda_{\max}^2 ||z||^2 \label{eq:scalar-mul-1}
	\end{align}
	
	А норма $||z|| = (z, z) = \sum\limits_i z_i^2$. Тогда справедливо утверждение:
	
	\begin{equation}
		\label{eq:norm-az-leq}
		||A_z|| \leq |\lambda_{\max}| ||z||
	\end{equation}
	
	
	

	
\end{proof}

Рассмотрим случай самосопряженного оператора $A_h$. Пусть $A_h = A^{*}_n$, тогда норма самосопряженного оператора 